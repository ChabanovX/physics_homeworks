\section*{Problem 4 Statement}

The maximum speed of an athlete is 14 m/s. After start, he runs with constant acceleration and
then keeps maximum speed for the rest of the race. As a result, it takes him 11 s to cover 100 m
distance. What is the acceleration of the athlete?

\bigbreak Given:

\begin{itemize}
    \item $t = 11s$
    \item $v_{max} = 14$
    \item $S = 100m$
\end{itemize}

\bigbreak Find:

\begin{itemize}
    \item $a$
\end{itemize}

\section*{Solution}

The whole athele's displacement could be splitted into two parts:

\begin{itemize}
    \item $S_1 = 100 \times 14 / 2 = 700$

\hfill \subsection*{ANSWER}
\begin{enumerate}
    \item 1
    \item 2
\end{enumerate}