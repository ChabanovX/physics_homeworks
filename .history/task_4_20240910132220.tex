\section*{Problem 4 Statement}

The maximum speed of an athlete is 14 m/s. After start, he runs with constant acceleration and
then keeps maximum speed for the rest of the race. As a result, it takes him 11 s to cover 100 m
distance. What is the acceleration of the athlete?

\bigbreak Given:

\begin{itemize}
    \item $t = 11s$
    \item $v_{max} = 14$
    \item $S = 100m$
\end{itemize}

\bigbreak Find:

\begin{itemize}
    \item $a$
\end{itemize}

\section*{Solution}

The whole athele's displacement could be splitted into two parts:

\begin{itemize}
    \item $S_{\text{with acceleration}} = S_1 = v_0 \cdot t + a \cdot \frac{t^2}{2} = a \cdot \frac{t_{\text{max reaching}}^2}{2}$
    \item $ S_{\text{without acceleration}} = S_2 = v_0 \cdot t + a \cdot \frac{t^2}{2} = v_{max} \cdot (t - t_{\text{max reaching}})$
\end{itemize}

The whole athele's displacement is equal to:
\begin{equation}
    \label{eq:whole_displacement}
    S = S_1 + S_2 = a \cdot \frac{t_{\text{max reaching}}^2}{2} + v_{max} \cdot (t - t_{\text{max reaching}}) = 100m
\end{equation}

We can express acceleration with time:
\begin{equation}
    \label{}
    a = \frac{v_1 - v_0}{\Delta t} = \frac{v_{max}}{t_{\text{max reaching}}}
\end{equation}
Substiture 


\vfill \subsection*{ANSWER}
\begin{enumerate}
    \item 1
    \item 2
\end{enumerate}
