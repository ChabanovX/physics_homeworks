\documentclass{article}
\usepackage{amsmath}
\usepackage{amsfonts}
\usepackage{amssymb}
\usepackage{graphicx}
\usepackage{tikz}

\begin{document}

\title{Solution to Problem 1}
\author{}
\date{}
\maketitle

\section*{Problem Statement}
The figure below shows the time dependence of velocity. Do the following:

\begin{enumerate}
    \item Plot the acceleration and displacement with respect to time. Assume the initial coordinate is $x(0) = 0$ m.
    \item Determine the displacement and the average velocity over the time interval $[t_1, t_3]$.
\end{enumerate}

Given: $t_1 = 4$ s, $t_2 = 10$ s, $t_3 = 18$ s.

\section*{Solution}

\subsection*{1. Acceleration and Displacement Plots}

\subsubsection*{Acceleration}
Acceleration is the derivative of velocity with respect to time. The velocity function can be described in three intervals:

\[
v(t) =
\begin{cases}
\frac{v_1}{t_1} t, & \text{for } 0 \leq t \leq t_1 \\
v_1, & \text{for } t_1 \leq t \leq t_2 \\
-\frac{v_1}{t_3 - t_2} (t - t_2) + v_1, & \text{for } t_2 \leq t \leq t_3
\end{cases}
\]

The acceleration is the derivative of $v(t)$ with respect to $t$:

\[
a(t) =
\begin{cases}
\frac{v_1}{t_1}, & \text{for } 0 \leq t \leq t_1 \\
0, & \text{for } t_1 \leq t \leq t_2 \\
-\frac{v_1}{t_3 - t_2}, & \text{for } t_2 \leq t \leq t_3
\end{cases}
\]

\begin{center}
\begin{tikzpicture}
\draw[->] (0,0) -- (10,0) node[right] {$t(s)$};
\draw[->] (0,0) -- (0,4) node[above] {$a(m/s^2)$};
\draw[thick] (0,0) -- (4,3) node[above right] {$\frac{v_1}{t_1}$};
\draw[thick] (4,0) -- (10,0) node[above right] {$0$};
\draw[thick] (10,0) -- (18,-1) node[below right] {$-\frac{v_1}{t_3 - t_2}$};
\end{tikzpicture}
\end{center}

\subsubsection*{Displacement}
The displacement is the integral of the velocity function with respect to time:

\[
x(t) = \int v(t) \, dt
\]

For the given intervals:

\[
x(t) =
\begin{cases}
\frac{v_1}{2t_1} t^2, & \text{for } 0 \leq t \leq t_1 \\
v_1(t - t_1) + x(t_1), & \text{for } t_1 \leq t \leq t_2 \\
-\frac{v_1}{2(t_3 - t_2)}(t - t_2)^2 + v_1(t - t_2) + x(t_2), & \text{for } t_2 \leq t \leq t_3
\end{cases}
\]

The displacement graph is plotted below:

\begin{center}
\begin{tikzpicture}
\draw[->] (0,0) -- (10,0) node[right] {$t(s)$};
\draw[->] (0,0) -- (0,4) node[above] {$x(m)$};
\draw[thick] (0,0) parabola (4,4) node[above right] {$\frac{v_1}{2t_1}t^2$};
\draw[thick] (4,4) -- (10,6) node[above right] {$v_1(t-t_1) + x(t_1)$};
\draw[thick] (10,6) parabola bend (14,7) (18,4) node[below right] {$-\frac{v_1}{2(t_3-t_2)}(t-t_2)^2 + v_1(t-t_2) + x(t_2)$};
\end{tikzpicture}
\end{center}

\subsection*{2. Displacement and Average Velocity}

The displacement from $t_1$ to $t_3$ can be found by integrating the velocity over this interval:

\[
\Delta x = \int_{t_1}^{t_3} v(t) \, dt
\]

For $t_1 = 4$ s, $t_2 = 10$ s, and $t_3 = 18$ s:

\[
\Delta x_{[t_1, t_2]} = v_1(t_2 - t_1) = 4(10 - 4) = 24 \text{ m}
\]

\[
\Delta x_{[t_2, t_3]} = \frac{1}{2} \times (v_1 \times (t_3 - t_2)) = \frac{1}{2} \times (4 \times 8) = 16 \text{ m}
\]

Thus:

\[
\Delta x = \Delta x_{[t_1, t_2]} + \Delta x_{[t_2, t_3]} = 24 \text{ m} + 16 \text{ m} = 40 \text{ m}
\]

The average velocity over the interval $[t_1, t_3]$ is:

\[
v_{\text{avg}} = \frac{\Delta x}{t_3 - t_1} = \frac{40}{18 - 4} = \frac{40}{14} \approx 2.86 \text{ m/s}
\]

\end{document}