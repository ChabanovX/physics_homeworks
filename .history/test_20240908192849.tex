\documentclass{article}
\usepackage{amsmath}

\begin{document}

\section*{Problem 2}
A ball is thrown horizontally from a height of $20 \, \text{m}$ and hits the ground with a speed that is three times its initial speed. What is the initial speed? Assume that there is no air drag and therefore acceleration in the horizontal direction is zero.

\section*{Solution}

Given:
\begin{itemize}
    \item Initial height: $h = 20 \, \text{m}$
    \item Final speed: $v_f = 3v_0$
    \item Acceleration in the horizontal direction is zero.
\end{itemize}

### Step 1: Vertical motion analysis

The time it takes for the ball to hit the ground can be determined using the following kinematic equation for vertical motion:
\[
h = \frac{1}{2} g t^2
\]
where \( g \) is the acceleration due to gravity \( (g = 9.8 \, \text{m/s}^2) \). 

Solving for time \( t \):
\[
t = \sqrt{\frac{2h}{g}} = \sqrt{\frac{2 \times 20 \, \text{m}}{9.8 \, \text{m/s}^2}} \approx 2.02 \, \text{s}
\]

### Step 2: Horizontal motion analysis

The horizontal speed remains constant throughout the motion since there is no horizontal acceleration. The initial horizontal speed is \( v_0 \), and the final horizontal speed is \( v_f = 3v_0 \).

### Step 3: Final speed calculation

The final speed \( v_f \) is the vector sum of the horizontal and vertical components of the velocity. At the moment the ball hits the ground:
\[
v_f = \sqrt{v_0^2 + v_y^2}
\]
where \( v_y \) is the vertical speed when the ball hits the ground. \( v_y \) can be found using:
\[
v_y = g t
\]
Substitute the value of \( t \):
\[
v_y = 9.8 \, \text{m/s}^2 \times 2.02 \, \text{s} \approx 19.8 \, \text{m/s}
\]

### Step 4: Solve for \( v_0 \)

Given that \( v_f = 3v_0 \), we can write:
\[
(3v_0)^2 = v_0^2 + v_y^2
\]
Expanding and rearranging:
\[
9v_0^2 = v_0^2 + v_y^2
\]
\[
8v_0^2 = v_y^2
\]
\[
v_0 = \frac{v_y}{\sqrt{8}} = \frac{19.8 \, \text{m/s}}{\sqrt{8}} \approx 6.99 \, \text{m/s}
\]

### Final Answer:
The initial speed of the ball is approximately \( v_0 = 7 \, \text{m/s} \).

\end{document}