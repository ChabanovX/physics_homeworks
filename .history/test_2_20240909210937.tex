\documentclass{article}
\usepackage{amsmath}

\begin{document}

\section*{Solution}

Let's denote:
\begin{itemize}
    \item \( v_b \) as the speed of the boat relative to the water.
    \item \( v_r \) as the speed of the river.
    \item The time elapsed since the bottle was dropped is \( t \) hours.
\end{itemize}

\subsection*{Upstream and Downstream Analysis}

When the fisherman drops the bottle and moves upstream, the boat's effective speed is \( v_b - v_r \). The bottle will drift downstream with the river at \( v_r \).

\textbf{Distance covered by the boat in 0.5 hours:}
\[
\text{Distance covered} = (v_b - v_r) \times 0.5 \, \text{km}
\]

\textbf{Distance to the bottle after 0.5 hours:}
\[
\text{Total distance to the bottle} = 5 + 0.5(v_b - v_r) \, \text{km}
\]

The fisherman now moves downstream to the bottle at a speed \( v_b + v_r \).

\subsection*{Time Taken to Reach the Bottle}

The time taken by the fisherman to reach the bottle after turning back:
\[
\text{Time taken} = \frac{5 + 0.5(v_b - v_r)}{v_b + v_r}
\]

Since the time taken to travel upstream and downstream is equal, we have:
\[
\text{Total time} = t = \frac{0.5}{(v_b - v_r)} + \frac{5 + 0.5(v_b - v_r)}{v_b + v_r}
\]

\subsection*{Simplifying the Equation}

Using the fact that the bottle drifts with the river speed \( v_r \) and the total distance covered is 5 km:
\[
v_r t = 5 \, \text{km}
\]

Solving this equation for \( v_r \) will yield the speed of the river.

\end{document}