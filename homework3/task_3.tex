A small mass slides down with zero initial velocity from the top of a smooth hill of height H.
The foot of the hill has a portion of horizontal surface 
before the vertical cliff of height h.


\bigbreak Given: 
\begin{itemize}
    \item Mass starts from rest at the top of a smooth hill of height \( H \).
    \item At the bottom of the hill, there's a horizontal surface leading to a vertical cliff of height \( h \).
    \item At the bottom of the hill, there's a horizontal surface leading to a vertical cliff of height \( h \).
\end{itemize}

\bigbreak Find: 
\begin{itemize}
    \item What must be the height of the horizontal portion $(h/H = ?)$ to ensure the maximum distance $s$
    covered by the flying mass?
    \item What the maximum distance $s$
\end{itemize}

\section*{Solution} \subsection*{Horizontal portion}

Since the hill is smooth (no friction), mechanical energy is conserved during the descent.
\bigbreak Potential Energy at the Top:
  \[
  E_{\text{top}} = m g H
  \]

\smallbreak Potential Energy at Height \( h \):
  \[
  E_{\text{bottom}} = m g h
  \]

\smallbreak Kinetic Energy at the Bottom:
  \[
  K_{\text{bottom}} = \frac{1}{2} m v_0^2
  \]

\smallbreak Apply Energy Conservation:
\[
E_{\text{top}} = E_{\text{bottom}} + K_{\text{bottom}}
\]
\[
m g H = m g h + \frac{1}{2} m v_0^2
\]

\smallbreak Solving for \( v_0 \):
$$ \frac{1}{2} m v_0^2 = m g (H - h) $$
$$ v_0 = \sqrt{2 g (H - h)} $$


\bigbreak Now let's calculate time of flight off the cliff.
\smallbreak Vertical Motion:
  \[
  y = v_{y0} t + \frac{1}{2} a t^2
  \] \smallbreak 
  Since the mass moves horizontally off the cliff:
  \[
  v_{y0} = 0, \quad a = -g, \quad y = -h
  \] \smallbreak 
  So:
  $$
  -h = 0 - \frac{1}{2} g t^2 $$
  $$ t = \sqrt{\frac{2 h}{g}}
  $$

\bigbreak Proceed with Horizontal Distance \( s \)
  \begin{equation}
    \label{eq:maxdist}
    s = v_0 t = \sqrt{2 g (H - h)} * \sqrt{\frac{2 h}{g}} = 2 \sqrt{h (H - h)}
  \end{equation}

We need to maximize:
\[
s(h) = 2 \sqrt{h (H - h)}
\]

Let's define:
\[
f(h) = h (H - h) = h H - h^2
\]

To find the maximum of \( f(h) \), take the derivative with respect to \( h \) and set it to zero:
$$ f'(h) = H - 2 h = 0 $$
$$ 2 h = H $$
$$ \boxed{h = \frac{H}{2}} $$

\subsection*{Maximum distance}
Just simply substitute $h = \frac{H}{2}$ into \ref{eq:maxdist}.
\[
s_{\text{max}} = 2 \sqrt{\left( \frac{H}{2} \right) \left( H - \frac{H}{2} \right)} = 2 \sqrt{\left( \frac{H}{2} \right) \left( \frac{H}{2} \right)} = 2 \left( \frac{H}{2} \right) = \boxed{H}
\]

\vfill \subsection*{ANSWER}
\begin{itemize}
    \item $\frac{H}{2}$
    \item $H$
\end{itemize}