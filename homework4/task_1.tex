Two boats were slowly moving by inertia (drag forces exerted by water are negligibly small) along parallel courses towards each other. When the boats reached each other, the load of 25 kg was carefully reloaded from first boat to second one. After that, the second loaded boat stopped, first boat continued moving with velocity of 8 m/s. \textbf{What were the initial velocities of the boats (m/s, round to 1 decimal place), if the mass of second boat before reloading was 1 ton?}

\bigbreak Given:
\begin{itemize}
    \item $m$ = 25kg
    \item $M_2$ = 1ton = 1000kg
    \item $M_2'$ = $M_2$ + m = 1025kg ($'$ means after loading)
    \item $v_1'$ = 8m/s
    \item $v_2'$ = 0m/s
\end{itemize}

\bigbreak Find:
\begin{itemize}
    \item $v_0$, $v_1$
\end{itemize}

\section*{Solution}
We will abstract the event as relationship between the load and the second boat.
The load was moving on the first boat, hence, had its speed. Then the impulse was created due to reloading.
\textbf{Proceed with the Conservation of the Momentum:}
$$ m v_1 + M_2 v_2 = (m + M_2) v_2' $$
Since the second boat stopped:
\[ m v_1 + M_2 v_2 = 0 \]
Since boats are moving parallel courses towards each other, we should point they have opposite signs.
\textbf{Let's then consider $v_2$ negative.}
\[ m v_1 - M_2 v_2 = 0 \]
Rearrange:
\[ m v_1 = M_2 v_2 \]
\subsubsection*{Little Conversation}
At this point we should consider what truly happened during the event. The load from the first boat was carefully reloaded.
It implies that the losses in kinetic are approaching zero. 
From this consideration it is obvious that after the impulse the first boat did not change its speed.  $$ \textit{Hence, } v_1 = v_1' $$
Substitute $ v_1 = 8$:
$$ v_2 = \frac{mv_1}{M_2} = \frac{25 \cdot 8}{1000} = \boxed{0.2 \, m/s} $$


\vfill \subsection*{ANSWER}
\begin{itemize}
    \item $v_1 = 8 \, m/s $
    \item $v_2 = 0.2 \, m/s$
\end{itemize}