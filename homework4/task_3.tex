Three blocks are placed on a frictionless surface, with blocks of masses $m_2$ and $m_3$ being initially at rest. Block 1 is moving with a velocity of $v_{1i} = 10 \, \text{m/s}$. It then collides with block 2 (left figure), which then collides with block 3 (right figure). The third block has mass $m_3 = 6 \, \text{kg}$. After the second collision, block 2 becomes stationary again, while block 3 has the velocity of $v_{3f} = 5 \, \text{m/s}$. 
\textbf{Assuming all collisions to be elastic, find the final velocity $v_{1f}$ of block 1 (m/s, round to 1 decimal place).}

\bigbreak Given:
\begin{itemize}
    \item \( m_3 = 6\,\text{kg} \)
    \item \( v_{3f} = 5\,\text{m/s} \)
    \item \( v_{2f} = 0\,\text{m/s} \)
\end{itemize}

\bigbreak Find:
\begin{itemize}
    \item $v_{1f}$
\end{itemize}

\section*{Solution} \subsection*{Second collision}
Let's examine the event of the collision of second and third blocks a bit closer.
\textbf{Conservation of Momentum:}
\[
m_2 v_{2i} + m_3 v_{3i} = m_2 v_{2f} + m_3 v_{3f}
\]
Since third block was resting and afterwards second block rest:
\[
m_2 v_{2i} + 0 = 0 + m_3 v_{3f}
\]
\begin{equation}
    m_2 v_{2i} = 6\,\text{kg} \times 5\,\text{m/s} = 30\,\text{kg}\cdot\text{m/s}
    \label{impulse}
\end{equation}
Also known, that kinetic energy was also conserved.
\[
\frac{1}{2} m_2 v_{2i}^2 + \frac{1}{2} m_3 v_{3i}^2 = \frac{1}{2} m_2 v_{2f}^2 + \frac{1}{2} m_3 v_{3f}^2
\]
\[
\frac{1}{2} m_2 v_{2i}^2 + 0 = 0 + \frac{1}{2} m_3 v_{3f}^2
\]
\begin{equation}
    \frac{1}{2} m_2 v_{2i}^2 = \frac{1}{2} \times 6\,\text{kg} \times (5\,\text{m/s})^2 = 75\,\text{J}
    \label{kinetic_cons}
\end{equation}
\textbf{Combine:}
\[
v_{2i} = \frac{30\,\text{kg}\cdot\text{m/s}}{{m_2}}
\quad \text{from \ref{impulse}. Putting in \ref{kinetic_cons} produces:}
\]
\[
\frac{1}{2} m_2 \left( \frac{30}{m_2} \right)^2 = 75
\]

\[
\implies m_2 = 6\,\text{kg}
\]
Therefore, \( v_{2i} = \frac{30}{6} = 5\,\text{m/s} \)

\subsection*{First collision}
The procedure is the same. \textbf{Conservation of Momentum:}
    \[
    m_1 v_{1i} = m_1 v_{1f} + m_2 v_{2i}
    \]
    Plug in known values:
    \[
    m_1 \times 10 = m_1 v_{1f} + 6\,\text{kg} \times 5\,\text{m/s}
    \]
    Simplify:
    \begin{equation}
        m_1 (10 - v_{1f}) = 30\,\text{kg}\cdot\text{m/s}
        \label{secimp}
    \end{equation}
\textbf{Conservation of Kinetic Energy:}
    \[
    \frac{1}{2} m_1 v_{1i}^2 = \frac{1}{2} m_1 v_{1f}^2 + \frac{1}{2} m_2 v_{2i}^2
    \]
    Plug in known values:
    \[
    \frac{1}{2} m_1 (10)^2 = \frac{1}{2} m_1 v_{1f}^2 + \frac{1}{2} \times 6\,\text{kg} \times (5\,\text{m/s})^2
    \]
    Simplify:
    \begin{equation}
        m_1 (100 - v_{1f}^2) = 150\,\text{J} \quad \text{(2)}
        \label{seckin}
    \end{equation}
Proceed:
    \[
    m_1 = \frac{30}{10 - v_{1f}} \quad \text{from \ref{secimp}. Substitute in \ref{seckin}}
    \]
    
    \[
    \left( \frac{30}{10 - v_{1f}} \right) (100 - v_{1f}^2) = 150
    \]
    \[
    v_{1f}^2 - 5 v_{1f} - 50 = 0
    \]
    \[
    v_{1f} = \frac{5 \pm \sqrt{225}}{2} = \boxed{\frac{5 \pm 15}{2}}
    \]
Positive value does not make physical sense. \( v_{1f} = -5\,\text{m/s} \)

\vfill \subsection*{ANSWER}
\begin{itemize}
    \item $-5\,\text{m/s}$
\end{itemize}



